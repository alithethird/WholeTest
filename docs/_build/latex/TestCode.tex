%% Generated by Sphinx.
\def\sphinxdocclass{report}
\documentclass[letterpaper,10pt,english]{sphinxmanual}
\ifdefined\pdfpxdimen
   \let\sphinxpxdimen\pdfpxdimen\else\newdimen\sphinxpxdimen
\fi \sphinxpxdimen=.75bp\relax

\PassOptionsToPackage{warn}{textcomp}
\usepackage[utf8]{inputenc}
\ifdefined\DeclareUnicodeCharacter
% support both utf8 and utf8x syntaxes
\edef\sphinxdqmaybe{\ifdefined\DeclareUnicodeCharacterAsOptional\string"\fi}
  \DeclareUnicodeCharacter{\sphinxdqmaybe00A0}{\nobreakspace}
  \DeclareUnicodeCharacter{\sphinxdqmaybe2500}{\sphinxunichar{2500}}
  \DeclareUnicodeCharacter{\sphinxdqmaybe2502}{\sphinxunichar{2502}}
  \DeclareUnicodeCharacter{\sphinxdqmaybe2514}{\sphinxunichar{2514}}
  \DeclareUnicodeCharacter{\sphinxdqmaybe251C}{\sphinxunichar{251C}}
  \DeclareUnicodeCharacter{\sphinxdqmaybe2572}{\textbackslash}
\fi
\usepackage{cmap}
\usepackage[T1]{fontenc}
\usepackage{amsmath,amssymb,amstext}
\usepackage{babel}
\usepackage{times}
\usepackage[Bjarne]{fncychap}
\usepackage{sphinx}

\fvset{fontsize=\small}
\usepackage{geometry}

% Include hyperref last.
\usepackage{hyperref}
% Fix anchor placement for figures with captions.
\usepackage{hypcap}% it must be loaded after hyperref.
% Set up styles of URL: it should be placed after hyperref.
\urlstyle{same}
\addto\captionsenglish{\renewcommand{\contentsname}{Contents:}}

\addto\captionsenglish{\renewcommand{\figurename}{Fig.\@ }}
\makeatletter
\def\fnum@figure{\figurename\thefigure{}}
\makeatother
\addto\captionsenglish{\renewcommand{\tablename}{Table }}
\makeatletter
\def\fnum@table{\tablename\thetable{}}
\makeatother
\addto\captionsenglish{\renewcommand{\literalblockname}{Listing}}

\addto\captionsenglish{\renewcommand{\literalblockcontinuedname}{continued from previous page}}
\addto\captionsenglish{\renewcommand{\literalblockcontinuesname}{continues on next page}}
\addto\captionsenglish{\renewcommand{\sphinxnonalphabeticalgroupname}{Non-alphabetical}}
\addto\captionsenglish{\renewcommand{\sphinxsymbolsname}{Symbols}}
\addto\captionsenglish{\renewcommand{\sphinxnumbersname}{Numbers}}

\addto\extrasenglish{\def\pageautorefname{page}}

\setcounter{tocdepth}{1}



\title{Test Code Documentation}
\date{Jun 24, 2022}
\release{1.0.0}
\author{Ali Ugur}
\newcommand{\sphinxlogo}{\vbox{}}
\renewcommand{\releasename}{Release}
\makeindex
\begin{document}

\pagestyle{empty}
\sphinxmaketitle
\pagestyle{plain}
\sphinxtableofcontents
\pagestyle{normal}
\phantomsection\label{\detokenize{index::doc}}



\chapter{Control Hub Main}
\label{\detokenize{index:module-Control_Hub.control_hub.control_hub}}\label{\detokenize{index:control-hub-main}}\index{Control\_Hub.control\_hub.control\_hub (module)@\spxentry{Control\_Hub.control\_hub.control\_hub}\spxextra{module}}
This python file is for Control Hub to achieve communication between Test Robot and UI
\index{ClientNode (class in Control\_Hub.control\_hub.control\_hub)@\spxentry{ClientNode}\spxextra{class in Control\_Hub.control\_hub.control\_hub}}

\begin{fulllineitems}
\phantomsection\label{\detokenize{index:Control_Hub.control_hub.control_hub.ClientNode}}\pysigline{\sphinxbfcode{\sphinxupquote{class }}\sphinxcode{\sphinxupquote{Control\_Hub.control\_hub.control\_hub.}}\sphinxbfcode{\sphinxupquote{ClientNode}}}
This is the main Control Hub communication class
\begin{quote}\begin{description}
\item[{Parameters}] \leavevmode
\sphinxstyleliteralstrong{\sphinxupquote{Node}} (\sphinxstyleliteralemphasis{\sphinxupquote{rclpy.Node}}) \textendash{} Main ROS2 node of the Control Hub

\end{description}\end{quote}
\index{callback\_button\_state() (Control\_Hub.control\_hub.control\_hub.ClientNode method)@\spxentry{callback\_button\_state()}\spxextra{Control\_Hub.control\_hub.control\_hub.ClientNode method}}

\begin{fulllineitems}
\phantomsection\label{\detokenize{index:Control_Hub.control_hub.control_hub.ClientNode.callback_button_state}}\pysiglinewithargsret{\sphinxbfcode{\sphinxupquote{callback\_button\_state}}}{\emph{request}, \emph{response}}{}
This callback function runs when a button read request comes from UI
\begin{quote}\begin{description}
\item[{Parameters}] \leavevmode\begin{itemize}
\item {} 
\sphinxstyleliteralstrong{\sphinxupquote{request}} (\sphinxstyleliteralemphasis{\sphinxupquote{ROBOT.goal}}) \textendash{} request coming from UI to read button state

\item {} 
\sphinxstyleliteralstrong{\sphinxupquote{response}} (\sphinxstyleliteralemphasis{\sphinxupquote{example\_interfaces.srv.Trigger}}) \textendash{} response of button read value

\end{itemize}

\item[{Returns}] \leavevmode
response

\item[{Return type}] \leavevmode
example\_interfaces.srv.Trigger

\end{description}\end{quote}

\end{fulllineitems}

\index{callback\_ui\_subscription() (Control\_Hub.control\_hub.control\_hub.ClientNode method)@\spxentry{callback\_ui\_subscription()}\spxextra{Control\_Hub.control\_hub.control\_hub.ClientNode method}}

\begin{fulllineitems}
\phantomsection\label{\detokenize{index:Control_Hub.control_hub.control_hub.ClientNode.callback_ui_subscription}}\pysiglinewithargsret{\sphinxbfcode{\sphinxupquote{callback\_ui\_subscription}}}{\emph{msg}}{}
This callback function is run when
a message comes from UI through topic.
When a message is received that message is
send to the Test Robot to activate or deactivate
the LED.
\begin{quote}\begin{description}
\item[{Parameters}] \leavevmode
\sphinxstyleliteralstrong{\sphinxupquote{msg}} (\sphinxstyleliteralemphasis{\sphinxupquote{string}}) \textendash{} input field data coming from UI

\end{description}\end{quote}

\end{fulllineitems}

\index{get\_result\_callback() (Control\_Hub.control\_hub.control\_hub.ClientNode method)@\spxentry{get\_result\_callback()}\spxextra{Control\_Hub.control\_hub.control\_hub.ClientNode method}}

\begin{fulllineitems}
\phantomsection\label{\detokenize{index:Control_Hub.control_hub.control_hub.ClientNode.get_result_callback}}\pysiglinewithargsret{\sphinxbfcode{\sphinxupquote{get\_result\_callback}}}{\emph{future}}{}
This callback function runs when
the LED state response is acquired from Test Robot
\begin{quote}\begin{description}
\item[{Parameters}] \leavevmode
\sphinxstyleliteralstrong{\sphinxupquote{future}} (\sphinxstyleliteralemphasis{\sphinxupquote{future}}) \textendash{} future object to acquire result

\end{description}\end{quote}

\end{fulllineitems}

\index{goal\_response\_callback() (Control\_Hub.control\_hub.control\_hub.ClientNode method)@\spxentry{goal\_response\_callback()}\spxextra{Control\_Hub.control\_hub.control\_hub.ClientNode method}}

\begin{fulllineitems}
\phantomsection\label{\detokenize{index:Control_Hub.control_hub.control_hub.ClientNode.goal_response_callback}}\pysiglinewithargsret{\sphinxbfcode{\sphinxupquote{goal\_response\_callback}}}{\emph{future}}{}
This callback function runs when
the LED activation/deactivation request is accepted
by the Test Robot
\begin{quote}\begin{description}
\item[{Parameters}] \leavevmode
\sphinxstyleliteralstrong{\sphinxupquote{future}} (\sphinxstyleliteralemphasis{\sphinxupquote{future}}) \textendash{} future object to acquire goal acception

\end{description}\end{quote}

\end{fulllineitems}

\index{send\_goal() (Control\_Hub.control\_hub.control\_hub.ClientNode method)@\spxentry{send\_goal()}\spxextra{Control\_Hub.control\_hub.control\_hub.ClientNode method}}

\begin{fulllineitems}
\phantomsection\label{\detokenize{index:Control_Hub.control_hub.control_hub.ClientNode.send_goal}}\pysiglinewithargsret{\sphinxbfcode{\sphinxupquote{send\_goal}}}{\emph{gpio}}{}
This function is used to send the gpio data
to the Test Robot
\begin{quote}\begin{description}
\item[{Parameters}] \leavevmode
\sphinxstyleliteralstrong{\sphinxupquote{gpio}} (\sphinxstyleliteralemphasis{\sphinxupquote{string}}) \textendash{} input field data coming from UI

\end{description}\end{quote}

\end{fulllineitems}


\end{fulllineitems}

\index{main() (in module Control\_Hub.control\_hub.control\_hub)@\spxentry{main()}\spxextra{in module Control\_Hub.control\_hub.control\_hub}}

\begin{fulllineitems}
\phantomsection\label{\detokenize{index:Control_Hub.control_hub.control_hub.main}}\pysiglinewithargsret{\sphinxcode{\sphinxupquote{Control\_Hub.control\_hub.control\_hub.}}\sphinxbfcode{\sphinxupquote{main}}}{\emph{args=None}}{}
This is the main function that starts the ClientNode

\end{fulllineitems}



\chapter{Test Robot Main}
\label{\detokenize{index:module-Test_Robot.test_robot.test_robot.test_robot_server}}\label{\detokenize{index:test-robot-main}}\index{Test\_Robot.test\_robot.test\_robot.test\_robot\_server (module)@\spxentry{Test\_Robot.test\_robot.test\_robot.test\_robot\_server}\spxextra{module}}
This is an example code that is included in the pi\_gpio module for ROS2
This example code is modified to work well with Control Hub Test Code
\index{GPIOActionServer (class in Test\_Robot.test\_robot.test\_robot.test\_robot\_server)@\spxentry{GPIOActionServer}\spxextra{class in Test\_Robot.test\_robot.test\_robot.test\_robot\_server}}

\begin{fulllineitems}
\phantomsection\label{\detokenize{index:Test_Robot.test_robot.test_robot.test_robot_server.GPIOActionServer}}\pysigline{\sphinxbfcode{\sphinxupquote{class }}\sphinxcode{\sphinxupquote{Test\_Robot.test\_robot.test\_robot.test\_robot\_server.}}\sphinxbfcode{\sphinxupquote{GPIOActionServer}}}
This class is used to handle ROS2 Action server for
communication with Control Hub 
:param Node: Main node of the robot action server  
:type Node: rclpy.Node
\index{cancel\_callback() (Test\_Robot.test\_robot.test\_robot.test\_robot\_server.GPIOActionServer method)@\spxentry{cancel\_callback()}\spxextra{Test\_Robot.test\_robot.test\_robot.test\_robot\_server.GPIOActionServer method}}

\begin{fulllineitems}
\phantomsection\label{\detokenize{index:Test_Robot.test_robot.test_robot.test_robot_server.GPIOActionServer.cancel_callback}}\pysiglinewithargsret{\sphinxbfcode{\sphinxupquote{cancel\_callback}}}{\emph{goal}}{}
This callback runs when the current goal is cancelled
\begin{quote}\begin{description}
\item[{Parameters}] \leavevmode
\sphinxstyleliteralstrong{\sphinxupquote{goal}} (\sphinxstyleliteralemphasis{\sphinxupquote{ROBOT.goal}}) \textendash{} Goal coming from Control Hub

\item[{Returns}] \leavevmode
Cancel response acceptance

\item[{Return type}] \leavevmode
CancelResponse.ACCEPT

\end{description}\end{quote}

\end{fulllineitems}

\index{destroy() (Test\_Robot.test\_robot.test\_robot.test\_robot\_server.GPIOActionServer method)@\spxentry{destroy()}\spxextra{Test\_Robot.test\_robot.test\_robot.test\_robot\_server.GPIOActionServer method}}

\begin{fulllineitems}
\phantomsection\label{\detokenize{index:Test_Robot.test_robot.test_robot.test_robot_server.GPIOActionServer.destroy}}\pysiglinewithargsret{\sphinxbfcode{\sphinxupquote{destroy}}}{}{}
This function cleans the node

\end{fulllineitems}

\index{execute\_callback() (Test\_Robot.test\_robot.test\_robot.test\_robot\_server.GPIOActionServer method)@\spxentry{execute\_callback()}\spxextra{Test\_Robot.test\_robot.test\_robot.test\_robot\_server.GPIOActionServer method}}

\begin{fulllineitems}
\phantomsection\label{\detokenize{index:Test_Robot.test_robot.test_robot.test_robot_server.GPIOActionServer.execute_callback}}\pysiglinewithargsret{\sphinxbfcode{\sphinxupquote{execute\_callback}}}{\emph{goal\_handle}}{}
Executes the goal. The goal can be LED control
or reading the button state.
If the goal is LED control then 3 is sent back as result.
If the goal is reading button state then 0 or 1 is sent back as result.
\begin{quote}\begin{description}
\item[{Parameters}] \leavevmode
\sphinxstyleliteralstrong{\sphinxupquote{goal\_handle}} (\sphinxstyleliteralemphasis{\sphinxupquote{ROBOT\_interface}}) \textendash{} goal from Control Hub

\item[{Returns}] \leavevmode
result of execution

\item[{Return type}] \leavevmode
ROBOT.result

\end{description}\end{quote}

\end{fulllineitems}

\index{goal\_callback() (Test\_Robot.test\_robot.test\_robot.test\_robot\_server.GPIOActionServer method)@\spxentry{goal\_callback()}\spxextra{Test\_Robot.test\_robot.test\_robot.test\_robot\_server.GPIOActionServer method}}

\begin{fulllineitems}
\phantomsection\label{\detokenize{index:Test_Robot.test_robot.test_robot.test_robot_server.GPIOActionServer.goal_callback}}\pysiglinewithargsret{\sphinxbfcode{\sphinxupquote{goal\_callback}}}{\emph{goal\_request}}{}
This callback function is used to create
goal responce accept message
\begin{quote}\begin{description}
\item[{Parameters}] \leavevmode
\sphinxstyleliteralstrong{\sphinxupquote{goal\_request}} (\sphinxstyleliteralemphasis{\sphinxupquote{ROBOT.goal}}) \textendash{} Goal request coming from Control Hub

\item[{Returns}] \leavevmode
Goal acception

\item[{Return type}] \leavevmode
GoalResponse.ACCEPT

\end{description}\end{quote}

\end{fulllineitems}

\index{handle\_accepted\_callback() (Test\_Robot.test\_robot.test\_robot.test\_robot\_server.GPIOActionServer method)@\spxentry{handle\_accepted\_callback()}\spxextra{Test\_Robot.test\_robot.test\_robot.test\_robot\_server.GPIOActionServer method}}

\begin{fulllineitems}
\phantomsection\label{\detokenize{index:Test_Robot.test_robot.test_robot.test_robot_server.GPIOActionServer.handle_accepted_callback}}\pysiglinewithargsret{\sphinxbfcode{\sphinxupquote{handle\_accepted\_callback}}}{\emph{goal\_handle}}{}
This callback function handles the goal coming from Control Hub
\begin{quote}\begin{description}
\item[{Parameters}] \leavevmode
\sphinxstyleliteralstrong{\sphinxupquote{goal\_handle}} (\sphinxstyleliteralemphasis{\sphinxupquote{ROBOT\_interface}}) \textendash{} goal from Control Hub

\end{description}\end{quote}

\end{fulllineitems}


\end{fulllineitems}

\index{RaspberryPIGPIO (class in Test\_Robot.test\_robot.test\_robot.test\_robot\_server)@\spxentry{RaspberryPIGPIO}\spxextra{class in Test\_Robot.test\_robot.test\_robot.test\_robot\_server}}

\begin{fulllineitems}
\phantomsection\label{\detokenize{index:Test_Robot.test_robot.test_robot.test_robot_server.RaspberryPIGPIO}}\pysiglinewithargsret{\sphinxbfcode{\sphinxupquote{class }}\sphinxcode{\sphinxupquote{Test\_Robot.test\_robot.test\_robot.test\_robot\_server.}}\sphinxbfcode{\sphinxupquote{RaspberryPIGPIO}}}{\emph{pin\_id}, \emph{pin\_type}}{}
This class is used for controlling Raspberry Pi GPIO
\index{set\_pin() (Test\_Robot.test\_robot.test\_robot.test\_robot\_server.RaspberryPIGPIO method)@\spxentry{set\_pin()}\spxextra{Test\_Robot.test\_robot.test\_robot.test\_robot\_server.RaspberryPIGPIO method}}

\begin{fulllineitems}
\phantomsection\label{\detokenize{index:Test_Robot.test_robot.test_robot.test_robot_server.RaspberryPIGPIO.set_pin}}\pysiglinewithargsret{\sphinxbfcode{\sphinxupquote{set\_pin}}}{\emph{value}}{}
This function is used to set pins high or low 
:param value: pins value ro write
:type value: int

\end{fulllineitems}


\end{fulllineitems}



\chapter{Indices and tables}
\label{\detokenize{index:indices-and-tables}}\begin{itemize}
\item {} 
\DUrole{xref,std,std-ref}{genindex}

\item {} 
\DUrole{xref,std,std-ref}{modindex}

\item {} 
\DUrole{xref,std,std-ref}{search}

\end{itemize}


\renewcommand{\indexname}{Python Module Index}
\begin{sphinxtheindex}
\let\bigletter\sphinxstyleindexlettergroup
\bigletter{c}
\item\relax\sphinxstyleindexentry{Control\_Hub.control\_hub.control\_hub}\sphinxstyleindexpageref{index:\detokenize{module-Control_Hub.control_hub.control_hub}}
\indexspace
\bigletter{t}
\item\relax\sphinxstyleindexentry{Test\_Robot.test\_robot.test\_robot.test\_robot\_server}\sphinxstyleindexpageref{index:\detokenize{module-Test_Robot.test_robot.test_robot.test_robot_server}}
\end{sphinxtheindex}

\renewcommand{\indexname}{Index}
\printindex
\end{document}